\documentclass[11pt]{article}
\usepackage{enumitem}
\usepackage{xcolor}
\usepackage{fancyhdr}

%configure options for the hyperref package,
%which is used to add hyperlinks to documents
\usepackage{hyperref}
\hypersetup{
    colorlinks=true,
    linkcolor=blue,
    filecolor=magenta,
    urlcolor=blue
    }
\urlstyle{same}

%configure options for the page size
\usepackage[
%paperheight=16cm, paperwidth=12cm,% Set the height and width of the paper
includehead,
nomarginpar,% We don't want any margin paragraphs
%textwidth=10cm,% Set \textwidth to 10cm
%headheight=10mm,% Set \headheight to 10mm
]{geometry}


% Set the page style to "fancy"...
\pagestyle{fancy}
%... then configure it.
\fancyhead{} % clear all header fields
\fancyhead[L]{TSWR21 Project Report,  HT2024}

\title{Project Title}
\author{Team\_member 1, Team\_member 2, ...}
%\date{}
\begin{document}

\maketitle  % This command generates the title

\section{Introduction}

{\it Introduce the application in this section. After reading this section, the reader should understand what the application is about.}

\noindent Note:
\begin{itemize}
  \item In this template, all italic text should be removed and replaced with your own text (which should not be italic); the italic text is just a placeholder letting you know what to write there.

  \item If you use Figures or Tables, please make sure to give each one a caption and a figure/table number and refer to them from the main text!

  \item References should be provided where applicable.

\end{itemize}

\section{Application Description}

\subsection{Application System}

{\it Depict the fundamental structure of the application system in this section. After reading this section, the reader should understand the components of the application system and their relations and/or the interactions between them. }

\subsection{Technical Details}

{\it In this section you also should include the information of the data sources utilized in the application, the ontolog(ies) \cite{Guarino2009} used/developed if there are, particular method(s)/technique(s) if there are, and tool(s)/librar(ies) are used to develop the application.

\subsection{Application Usage}

{\it In this section you should present the functionalities of the application using several use cases. }

\section{Discussion}

Discuss the lessons you learned from the project and the possibilities and limitations of the theories and technologies learned from this course.

\begin{thebibliography}{99}

\bibitem{Guarino2009}
Guarino, Nicola, Daniel Oberle, and Steffen Staab. "What is an ontology?." {\it Handbook on ontologies} (2009): 1-17.

\end{thebibliography}



\end{document}
